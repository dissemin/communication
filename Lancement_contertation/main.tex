\documentclass[a4paper]{article}

\usepackage[french]{babel}
\usepackage[utf8x]{inputenc}
\usepackage{fullpage}
\usepackage{geometry}
\usepackage{amsmath}
\usepackage{graphicx}
\usepackage{url}
\usepackage{tabularx}
\usepackage[colorinlistoftodos]{todonotes}

\title{Concertation pour l'élaboration d'une politique institutionnelle en faveur de l'accès ouvert aux publications à l'ENS}
\author{Patricia Mirabile et Antonin Delpeuch \\
Représentants des élèves normaliens au conseil scientifique}

\begin{document}
\maketitle

\section*{Contexte}

Les publications scientifiques, et en particulier celles qui ont été financées par de l'argent public, constituent un patrimoine commun qui devrait être accessible à toute personne désireuse d'acquérir du savoir.

Pourtant,  les maisons d'éditions, en sciences comme en lettres, ont mis en place un système généralement très rentable par lequel elles acquièrent gratuitement, ou presque\footnote{cf \url{http://pablo.rauzy.name/openaccess/introduction.html}}, les articles scientifiques qu'elles revendent ensuite, sous la forme d'un accès ponctuel payant, pour les particuliers, ou d'abonnements hors de prix\footnote{cf \url{http://www.theguardian.com/science/2012/apr/24/harvard-university-journal-publishers-prices}}, pour les universités. Les auteur·e·s ne sont pas rétribué·e·s, les comités scientifiques de sélection ne sont pas ou peu payés, et les chercheur·e·s et étudiant·e·s voient leur accès au savoir limité (voire bloqué lorsqu'ils font partie d'universités trop pauvres pour payer les abonnements). Ceci représente non seulement un abus de biens publics, mais surtout un frein scandaleux au progrès de la science à travers la société et même le monde. 

\section*{Une politique pour l'accès ouvert à l'ENS}


Nous souhaitons donc que l'ENS encourage ses chercheurs et chercheuses à rendre leurs
articles disponibles gratuitement en ligne, à l'instar des nombreuses initiatives adoptées par des universités telles que universités de Liège, de Harvard, de Californie, et
beaucoup d'autres\footnote{cf \url{http://roarmap.eprints.org/}}.
Ceci devra bien sûr se faire en prenant en compte les spécificités propres aux différents domaines de recherche représentés à l'ENS.


\section*{Déroulement de la concertation}

Une concertation au sujet de la mise en place d'une politique institutionnelle à l'ENS est une étape nécessaire dans le développement du projet de l'accès ouvert aux publications, et nous souhaitons que le conseil scientifique assure sa visibilité et sa légitimité en s'en déclarant l'instigateur. L'association CAPSH\footnote{cf \url{http://association.dissem.in/}} (Comité pour l'Accessibilité aux Publications en Sciences et Humanités), formée de chercheurs et d'étudiants à (ou issus de) l'ENS, se propose ensuite de l'organiser, avec le soutien de l'administration de l'ENS. 

Cette concertation, qui sera l'occasion de tenir plusieurs réunions et conférences au sujet de l'accès ouvert, permettra de s'assurer que les avis de tout·e·s puissent être pris en compte dans l'élaboration d'une résolution finale, qui sera elle-même soumise au vote du conseil scientifique.

Nous invitons par ailleurs les personnes intéressées à s'inscrire à notre liste de diffusion en envoyant un mail à \url{annonces-inscription@dissem.in}.

\section*{Implémentation}

L'adoption formelle d'une politique est loin d'être suffisante pour obtenir des
progrès concrets en termes d'accès ouvert, elle nécessite aussi la mise en place des moyens institutionnels et techniques propres à favoriser la diffusion en accès ouvert des publications. Notre association développe actuellement une plateforme, Dissemin, conçue pour rendre la mise en libre accès des articles par les chercheurs rapide et facile. Nous avons déjà organisé un événement pour l'Open Access Week à l'ENS (le 22 octobre), lors duquel la plateforme a été lancée. Nous allons maintenant mettre en place une instance spécifique à l'ENS, où les chercheur·e·s de chaque département seront pré-inscrit·e·s. Lors de cette préinscription, les noms et identifiants CRI des chercheurs seront transmis à notre association. Pour assurer le respect des données personnelles, il est possible de s'opposer à cette préinscription en s'adressant à Émilie Pennetier (\url{emilie.pennetier@ens.fr}).

La plateforme est en ligne à l'adresse \url{http://dissem.in}. Pour plus d'informations sur notre projet, vous pouvez contacter l'équipe (\url{equipe@dissem.in}).


\end{document}
