\documentclass[30pt,a1paper]{tikzposter}

\usepackage{subcaption}

\title{How flexible are categorical models of meaning?}
\institute{University of Cambridge}
\author{Antonin Delpeuch}
\titlegraphic{Logo}
\usetheme{Simple}
% Simple
% Autumn
\begin{document}
    \maketitle
    \block{Why are we using tensors after all?}{
        \Large {

The category theory behind the distributional compositional model gives useful properties:
for instance, the spurious ambiguity in CCG has no impact on the final semantic representation of a sentence.
In practice however, implementations often depart from the original framework, because using
only tensors and linear maps does not always give the best results.

We can get both the theoretical guarantees
of category theory and the flexibility of alternate models of meaning using
the notion of \emph{free compact closed category}. }

    }
    \begin{columns}
        \column{0.5}
        \block{Monoidal}{
            Vertical ($\circ$) and horizontal ($\otimes$) composition of morphisms.
            \begin{minipage}{0.6\linewidth}
                \begin{tikzpicture}[scale=3,every node/.style={node distance=2.5cm,scale=0.7}]
\node[draw,rectangle] (fog) {$f \circ g$};
\node[above of=fog,node distance=3.5cm] (A1) {$A$};
\node[below of=fog,node distance=3.5cm] (C1) {$C$};
\draw (fog) -- (A1);
\draw[->] (fog) -- (C1);

\node[right of=fog] (eg1) {$=$};

\node[right of=eg1] (B2) {$B$};
\node[draw,rectangle,above of=B2,node distance=1.75cm] (f1) {$g$};
\node[above of=f1,node distance=2.04cm] (A2) {$A$};
\node[draw,rectangle,below of=B2,node distance=1.75cm] (g1) {$f$};
\node[below of=g1,node distance=2.04cm] (C2) {$C$};
\draw[->] (A2) -- (f1) -- (B2) -- (g1) -- (C2);

\begin{scope}[xshift=2.45cm]

\node[draw,rectangle] (fcg) {$f \otimes g$};
\node[above of=fcg,node distance=3.5cm] (A3) {$A \otimes C$};
\node[below of=fcg,node distance=3.5cm] (B3) {$B \otimes D$};
\draw[->] (A3) -- (fcg) -- (B3);

\node[right of=fcg] (eg2) {$=$};

\node[right of=eg2,draw,rectangle] (f2) {$f$}; 
\node[right of=f2,draw,rectangle] (g2) {$g$}; 
\node[above of=f2,node distance=3.5cm] (A4) {$A$};
\node[above of=g2,node distance=3.5cm] (C4) {$C$};
\node[below of=f2,node distance=3.5cm] (B4) {$B$};
\node[below of=g2,node distance=3.5cm] (D4) {$D$};

%\node[right of=A4,node distance=0.5cm] (tensor) {$\otimes$};
%\node[right of=B4,node distance=0.5cm] (tensor) {$\otimes$};
\draw[->] (A4) -- (f2) -- (B4);
\draw[->] (C4) -- (g2) -- (D4);

\end{scope}

\begin{scope}[xshift=4cm]
\end{scope}



\end{tikzpicture}

            \end{minipage}
            \begin{minipage}{0.25\linewidth}
                \begin{tikzpicture}[every node/.style={node distance=2cm}]
\node[draw,rectangle] (a1) {$f_2$};
%\node[right of=a1, node distance=0.8cm] {$\otimes$};
\node[draw,rectangle,right of= a1] (b1) {$g_2$};
\node[draw,rectangle,below of=a1] (c1) {$f_1$};
%\node[right of=c1, node distance=0.8cm] {$\otimes$};
\node[draw,rectangle,below of=b1] (d1) {$g_1$};
\node[above of=a1,node distance=1.8cm] (p1) {};
\node[above of=b1,node distance=1.8cm] (p2) {};
\node[below of=c1,node distance=1.8cm] (p3) {};
\node[below of=d1,node distance=1.8cm] (p4) {};
\draw (p1) -- (a1) -- (c1) -- (p3);
\draw (p2) -- (b1) -- (d1) -- (p4);
\node[left of=a1,node distance=1cm] {$\Big($};
\node[left of=c1,node distance=1cm] {$\Big($};
\node[right of=b1,node distance=1cm] {$\Big)$};
\node[right of=d1,node distance=1cm] {$\Big)$};

\node[right of=b1] (blank) {};
\node[below of=blank,node distance=1cm] (egal) {$=$};

\node[draw,rectangle,right of=blank] (a1) {$f_2$};
\node[draw,rectangle,right of= a1,node distance=2.7cm] (b1) {$g_2$};
\node[draw,rectangle,below of=a1] (c1) {$f_1$};
\node[draw,rectangle,below of=b1] (d1) {$g_1$};
\node[above of=a1,node distance=1.8cm] (p1) {};
\node[above of=b1,node distance=1.8cm] (p2) {};
\node[below of=c1,node distance=1.8cm] (p3) {};
\node[below of=d1,node distance=1.8cm] (p4) {};
\draw (p1) -- (a1) -- (c1) -- (p3);
\draw (p2) -- (b1) -- (d1) -- (p4);
\node[right of=egal,node distance=0.9cm] (p1) {\Large $\Bigg($};
\node[right of=p1,node distance=2.20cm] (p2) {\Large $\Bigg)$};
\node[right of=p1,node distance=2.8cm] (p3) {\Large $\Bigg($};
\node[right of=p3,node distance=2.1cm] (p4) {\Large $\Bigg)$};
%\node[right of=p2,node distance=0.25cm] (otimes) {$\otimes$};

\end{tikzpicture}


            \end{minipage}
        }
        \column{0.5}
        \block{Compact closed}{
            Monoidal, plus \emph{cups} $\epsilon$ and \emph{caps} $\eta$ for each object $A$ such that:
            \begin{tikzpicture}[scale=1.6,
every node/.style={%
text height=1.5ex,text depth=0.25ex,scale=0.7}]

\node at (0.0,0) (t2) {$A$};
\node at (1.7000000000000002,0) (t7) {$A^*$};
\node at (-0.7,-0.7) (name) {$\epsilon =$};
\node at (0.85,-1.4) (i) {$I$};

\draw[->] (0.0,-0.25) .. controls (0.0,-0.7217500000000001) and (0.37825,-1.1) .. (0.8500000000000001,-1.1) .. controls (1.3217500000000002,-1.1) and (1.7000000000000002,-0.7217500000000001) .. (1.7000000000000002,-0.25);

\begin{scope}[yshift=-3.2cm]

\node at (0.0,0) (t2) {$A^*$};
\node at (1.7000000000000002,0) (t7) {$A$};
\node at (-0.7,0.7) (name) {$\eta =$};
\node at (0.85,1.4) (i) {$I$};

\draw[->] (0.0,0.25) .. controls (0.0,0.7217500000000001) and (0.37825,1.1) .. (0.8500000000000001,1.1) .. controls (1.3217500000000002,1.1) and (1.7000000000000002,0.7217500000000001) .. (1.7000000000000002,0.25);
\end{scope}

\begin{scope}[xshift=3.2cm,yshift=-0.25cm]

    \node at (0,0) (t1) {$A$};
    \node at (0,-1.4) (t2) {$A$};

   \node at (1.7,-1.4) (t2) {$A^*$};
   \node at (3.4,-1.4) (t7) {$A$};
   \node at (2.55,0) (i) {$I$};

   \draw[->] (1.7,-1.15) .. controls (1.7,-0.68) and (2.08,-0.3) .. (2.55,-0.3) .. controls (3.0217500000000002,-0.3) and (3.4,-0.68) .. (3.4,-1.15);

   \draw[->] (0,-0.25) -- (0,-1.1);

\node at (0.85,-2.8) (i2) {$I$};

\draw[->] (0.0,-1.65) .. controls (0.0,-2.1217500000000001) and (0.37825,-2.5) .. (0.8500000000000001,-2.5) .. controls (1.3217500000000002,-2.5) and (1.7000000000000002,-2.1217500000000001) .. (1.7000000000000002,-1.65);

\draw[->] (3.4,-1.65) -- (3.4,-2.5);

   \node at (3.4,-2.75) (i3) {$A$};

   \node at (4.2,-1.4) (egal) {$=$};

   \node at (5,0) (tb) {$A$};
   \node at (5,-2.75) (tc) {$A$};

   \draw[->] (5,-0.25) -- (5,-2.5);

 \end{scope}

\begin{scope}[xshift=10cm, yshift=-3cm, yscale=-1]

    \node at (0,0) (t1) {$A^*$};
    \node at (0,-1.4) (t2) {$A^*$};

   \node at (1.7,-1.4) (t2) {$A$};
   \node at (3.4,-1.4) (t7) {$A^*$};
   \node at (2.55,0) (i) {$I$};

   \draw[->] (1.7,-1.15) .. controls (1.7,-0.68) and (2.08,-0.3) .. (2.55,-0.3) .. controls (3.0217500000000002,-0.3) and (3.4,-0.68) .. (3.4,-1.15);

   \draw[->] (0,-0.25) -- (0,-1.1);

\node at (0.85,-2.8) (i2) {$I$};

\draw[->] (0.0,-1.65) .. controls (0.0,-2.1217500000000001) and (0.37825,-2.5) .. (0.8500000000000001,-2.5) .. controls (1.3217500000000002,-2.5) and (1.7000000000000002,-2.1217500000000001) .. (1.7000000000000002,-1.65);

\draw[->] (3.4,-1.65) -- (3.4,-2.5);

   \node at (3.4,-2.75) (i3) {$A^*$};

   \node at (4.2,-1.4) (egal) {$=$};

   \node at (5,0) (tb) {$A^*$};
   \node at (5,-2.75) (tc) {$A^*$};

   \draw[->] (5,-0.25) -- (5,-2.5);

 \end{scope}

\end{tikzpicture}

        }
    \end{columns}
    \block{Recipe: turning a monoidal category $\mathcal{C}$ into a compact closed category $\mathcal{C'}$}{
        \begin{minipage}{0.46\linewidth}
        \begin{enumerate}
            \item Draw word meanings using morphisms from $\mathcal{C}$, using imaginary cups and caps to
                 introduce adjoints where needed
             \item Compose word meanings with the type reduction
                   and apply the yanking equalities in $\mathcal{C'}$.
            \item You get a diagram without cups and caps, hence a valid morphism in $\mathcal{C}$.
        \end{enumerate}
        \end{minipage}
        \begin{minipage}{0.46\linewidth}
            \hspace{3cm}
        \begin{tikzpicture}[scale=3,
every node/.style={%
text height=1.5ex,text depth=0.25ex,
scale=0.8}]
% Diagram generated by http://github.com/wetneb/MorozParser
\makeatletter


\pgfdeclareshape{triangle}{
    \inheritsavedanchors[from=rectangle]
    \inheritanchorborder[from=rectangle]
    \inheritanchor[from=rectangle]{center}
    \inheritanchor[from=rectangle]{north}
    \inheritanchor[from=rectangle]{south}
    \inheritanchor[from=rectangle]{east}
    \inheritanchor[from=rectangle]{southwest}
    \inheritanchor[from=rectangle]{southeast}

    \backgroundpath{
        \southwest \pgf@xb=\pgf@x \pgf@yb=\pgf@y
        \northeast \pgf@xa=\pgf@x \pgf@ya=\pgf@y
        \pgf@xc=0.5\pgf@xa \advance\pgf@xc by+0.5\pgf@xb

        \pgfpathmoveto{\pgfpoint{\pgf@xc}{\pgf@ya}}
        \pgfpathlineto{\pgfpoint{\pgf@xb}{\pgf@yb}}
        \pgfpathlineto{\pgfpoint{\pgf@xa}{\pgf@yb}}
        \pgfpathlineto{\pgfpoint{\pgf@xc}{\pgf@ya}}
    }
}

\tikzstyle{big-triangle}=[triangle,draw,inner sep=0pt,minimum width=0.5cm,minimum height=0.15cm]
\tikzstyle{vbig-triangle}=[big-triangle,scale=0.7]
\tikzstyle{vbig-triangle}=[big-triangle,scale=2]
\tikzstyle{mbig-triangle}=[big-triangle,scale=1.2]
\tikzstyle{small-triangle}=[big-triangle,scale=0.25]


\node at (0.0,0) (t2) {$N$};
\node at (0.0,1.5) (w0) {Pat};
\node at (1.5,0) (t8) {$N^r$ $S$ $N^l$};
\node at (1.5,1.5) (w1) {likes};
\node at (3.2,0) (t17) {$N$ $N^l$};
\node at (3.0,1.5) (w2) {female};
\node at (4.5,0) (t28) {$N$};
\node at (4.5,1.5) (w3) {octopuses};

%\node at (-1,2) (t) {};
%\node at (-1,0.05) (b) {};
%\draw[decorate,decoration={brace,amplitude=6pt}] (b) -- node[left] {\begin{tabular}{c} Word \\ semantics \end{tabular}} (t);

%\node at (-1,-0.05) (t) {};
%\node at (-1,-4) (b) {};
%\draw[decorate,decoration={brace,amplitude=6pt}] (b) -- node[left] {\begin{tabular}{c} Type \\ reduction \end{tabular}} (t);

% Curves below :
\draw[-latex] (0.0,-0.25) .. controls (0.0,-0.55525) and (0.24474999999999997,-0.8) .. (0.55,-0.8) .. controls (0.8552500000000001,-0.8) and (1.1,-0.55525) .. (1.1,-0.25);

\draw[latex-] (1.9,-0.25) .. controls (1.9,-0.5552499999999999) and (2.14475,-0.7999999999999998) .. (2.45,-0.7999999999999998) .. controls (2.75525,-0.7999999999999998) and (3,-0.5552499999999999) .. (3,-0.25);

\draw[-latex] (1.5,-0.25) -- (1.5,-1.1);

\draw[-latex] (4.5,-0.25) arc (0:-180:0.55cm);

\node at (1.5,-1.35) {$S$};

% Curves above:

\node[vbig-triangle] at (0.0,0.75) (fresh) {};
\node[scale=0.8] at ($(fresh)+(0,-0.05)$) {$p$};
\draw (0.0,0.25) -- (fresh);

\node[rectangle,draw,scale=0.8,minimum height=1.9cm,minimum width=1.5cm] at (1.5,0.62) (fresh) {$l$};
\draw (1.5,0.25) -- (fresh);
\draw (1.1,0.25) -- (1.1,0.82);
\draw (1.9,0.25) -- (1.9,0.82);
\draw[-latex] (1.1,0.82) arc (180:0:0.15cm);
\draw[-latex] (1.9,0.82) arc (0:180:0.15cm);

\node[vbig-triangle] at (4.5,0.75) (fresh) {};
\node[scale=0.8] at ($(fresh)+(0,-0.05)$) {$o$};
\draw (4.5,0.25) -- (fresh);

\begin{scope}[xshift=-3.9cm]
\node[rectangle,draw,minimum width=1.5cm,scale=0.8] at (6.9,0.6) (n) {$+$};
\draw[-latex] (n) -- ($(n.center)-(0,0.35)$);
\draw[latex-] ($(n.north)+(0.1,0)$) arc (180:0:0.15cm);
\draw ($(n.north)+(0.4,0)$) -- (7.3,0.25);
\node[vbig-triangle] at ($(n.north) +(-0.1,0.4)$) (tri) {};
\node[scale=0.8] at ($(tri)+(0,-0.05)$) {$f$};
\draw[latex-] ($(n.north)-(0.1,0)$) -- (tri);
\end{scope}

\begin{scope}[xshift=7cm,yshift=-1cm]
    \node at (-1.3,1) {{\Large $=$}};

\node at (0.0,0) {$S$};

\node[rectangle,draw,scale=0.8,minimum width=3cm] at (0,0.7) (fresh) {$l$};
\draw[latex-] (0,0.25) -- (fresh);

\node[vbig-triangle] at (0.1,1.8) (td1) {};
\node[scale=0.8] at ($(td1)+(0,-0.05)$) {$f$};
\node[vbig-triangle] at (0.4,1.8) (td2) {};
\node[scale=0.8] at ($(td2)+(0,-0.05)$) {$o$};

\node[rectangle,draw,scale=0.8,minimum width=2cm,fill=white] at (0.25,1.2) (n1) {$+$};

\draw[-latex] (n1) -- (n1 |- fresh.north);
\draw[-latex] (td1) -- (td1 |- n1.north);
\draw[-latex] (td2) -- (td2 |- n1.north);

\node[vbig-triangle] at (-0.25,1.2) (tg) {};
\node[scale=0.8] at ($(tg)+(0,-0.05)$) {$p$};
\draw[-latex] (tg) -- (tg |- fresh.north);

\end{scope}

\end{tikzpicture}


        \end{minipage}
    }

    \begin{columns}
        \column{0.5}
        \block{Cartesian}{
            The category of vector spaces with the cartesian product is not
            so it cannot be used as is for models of meaning,
            but has interesting properties.
        \begin{itemize}
            \item $\dim A \times B = \dim A + \dim B$: low dimensionality, word
                 representations are easier to learn
            \item Morphisms can be any maps, in particular
                 non-linear ones
            \item No compact closure: type reductions cannot be represented as morphisms
        \end{itemize}
        }
        \column{0.5}
        \block{Tensor}{
            This is the traditional category for models of meaning.
        \begin{itemize}
            \item $\dim A \otimes B = \dim A \times \dim B$: high dimensionality, word
                 representations are harder to learn
            \item Morphisms have to be linear, not monoidal otherwise
            \item Compact closure: type reductions can be represented as morphisms
        \end{itemize}
        }
    \end{columns}

    \block{}{Blocktext}


\end{document}
