\documentclass{scrartcl}
\usepackage[francais]{babel}
\usepackage[utf8]{inputenc}

\title{Compte-rendu de l'assemblée générale ordinaire de l'association C.A.P.S.H.}
\author{Antoine Amarilli}
\date{21 octobre 2017}

\hyphenation{ORCID}

\begin{document}
\maketitle

L'assemblée générale de l'association C.A.P.S.H.\ s'est tenue au cours de la
soirée du samedi 21 octobre 2017 au domicile d'Antoine Amarilli à Montrouge
(92), à partir de 20h.
% 
Sont présents :

\medskip

\begin{itemize}
\item Antoine Amarilli
\item Jean-Baptiste Bohuon
\item Antonin Delpeuch
\item Marie Farge
\item Marc Jeanmougin
\item Federico Leva
\item Patricia Mirabile
\end{itemize}

\medskip

Patricia Mirabile endosse la responsabilité de présidente de séance et Antoine
Amarilli celle de secrétaire de séance, ce que l'assemblée approuve à l'unanimité.
Les points suivants figurent à l'ordre
du jour :

\medskip

\begin{itemize}
\item Bilan moral
\item Bilan financier
\item Retour sur l'intégration d'OSF Preprints
\item Directions pour l'intégration ORCID
\item Réponse aux retours utilisateurs
\item Édition des données
\item Campagne Wikimédia et statistiques
\item Réunions périodiques et hackathons
\end{itemize}

\paragraph{Bilan moral.} Le bilan moral est présenté par Antonin Delpeuch. Il
fait brièvement état des principales actions entreprises par l'association au
cours de l'année écoulée. Il mentionne notamment l'intégration d'OSF Preprints,
dont toute l'assemblée se déclare très satisfaite, et mentionne l'investissement en temps
nécessaire pour encadrer d'autres projets de ce type.  L'assemblée générale
exprime également son appréciation pour la campagne menée par Federico Leva.

Est mentionnée la question du financement obtenu par le projet OpenIng, et de son
impact sur Dissemin. Il est noté que très peu d'informations sont disponibles pour nous
à ce stade : ampleur du financement, durée du projet, mode de développement (s'agirait-il
de contributions à la base de code principale ou d'un fork), embauche de développeurs, etc.

Il est décidé d'attendre une réponse à notre demande quant à l'écriture d'un billet de blog
sur le financement, mais de proposer une réunion en visioconférence avec les porteurs du
projet OpenIng pour échanger toutes les informations utiles sur le projet.

Le bilan moral est mis au vote et approuvé à l'unanimité.

\paragraph{Bilan financier.} Le bilan financier est présenté par Antoine
Amarilli. Il s'appuie sur des documents préalablement portés à la connaissance des
membres de l'AG par voie électronique, et dont la teneur suit.

Au 4 octobre, le solde du compte bancaire de l'association se porte à 5147.67~EUR.
Ses recettes mensuelles (dons récurrents) se portent à 15~EUR, et ses dépenses
mensuelles à 86.36~EUR. Ainsi, l'association perd-elle en moyenne 71.36~EUR par
mois, mais ses fonds courants permettent de financer ces dépenses pour les 6
prochaines années.
 
Sur la période courant du 1$^{\mathrm{er}}$ septembre 2016 au 31 août 2017, les
recettes de l'association se portent à 5990 EUR, dont 155~EUR de dons récurrents
(deux donateurs), une subvention du consortium Couperin (le 2017-12-20,
5400~EUR), et trois entrées ponctuelles : le 2016-10-07 (don de 300~EUR), le 2017-06-30
(don de 10~EUR), et le 2016-10-07 (prêt de 100~EUR par Antonin, non encore remboursé).

Les dépenses totales se montent à 1036.32~EUR de location de serveurs auprès de
Online.
 
En plus de ses recettes sur le compte en banque, l'association a reçu un
financement PSL Explore pour un montant de 15440~EUR. Ce financement a notamment
été utilisé pour financer un stage et des missions.

La question de la gestion du chéquier est posée, et il est proposé qu'il soit confié à Antoine.

La question de l'opportunité de soutenir davantage l'association est également soulevée, mais il est
décidé que, pour les dépenses courantes de l'association, les ressources actuelles suffisent
sans qu'il soit besoin de donner davantage. Au demeurant, la possibilité de poursuivre
d'autres collaborations avec le CCSD est toujours ouverte.

Il est discuté de réfléchir à financer la location des serveurs par le contrat PSL. En revanche, ceci
poserait sans doute des difficultés administratives. Il pourrait être possible de louer de nouveaux serveurs
avec PSL, peut-être chez un autre prestataire, et de déplacer les services sur ces serveurs, quitte à
les déplacer à nouveau à la fin du projet, mais ceci est perçu comme trop compliqué par rapport aux gains.

La question de financer un développeur est débattue. Les ressources de l'association ou de PSL ne
permettent pas un financement à temps plein sur une durée suffisante, mais il est envisagé de financer
à temps partiel, par exemple sur des vacations. Il n'est pas certain si ceci serait ouvert à des développeurs
n'exerçant pas d'activité principale, ou bien exerçant sous le régime de l'autoentrepreunariat. Différentes
solutions sont envisagées, pour du développement, ou de la maintenance, ou de la rédaction de documentation. Une question délicate est de 
savoir décrire les fonctions attendues, auditionner des candidats, et faire de la publicité autour d'une offre.
La question est envisagée mais sans être tranchée. Une autre question est de savoir si des universités,
par exemple le département informatique de l'ENS, pourraient signaler aux étudiants l'existence de
Dissemin, pour des stages ou pour des vacations. Ces questions ne sont pas tranchées, mais il est mentionné
qu'elles devraient être débattues avec les porteurs du projet OpenIng s'ils ont également l'intention de recruter du
personnel. La possibilité de faire appel à nouveau à Steph, dont la formation se termine cette année, est envisagée.

L'assemblée s'interroge quant au montant du financement PSL : quel montant est disponible (déduction
faite des frais de gestion), quel montant est engagé, et quel est le montant des
dépenses déjà effectuées ?
Marie indique qu'elle peut se renseigner à ce sujet.

Le bilan financier est mis au vote et approuvé à l'unanimité.

\paragraph{Retour sur l'intégration d'OSF Preprints.}
L'assemblée générale s'estime très satisfaite par le travail accompli pour l'intégration d'OSF Preprints.

\paragraph{Directions pour l'intégration ORCID.}
Antonin présente les sources d'insatisfaction avec l'intégration actuelle
d'ORCID, et les difficultés posées par le fait de ne pas avoir accès à leur API
payante. Il est rappelé l'échec de tentatives antérieures d'obtenir cet
abonnement, notamment via Couperin. La possibilité de poser la question à
OpenIng est mentionnée, mais un problème est le caractère temporaire de ce
projet. Il est discuté de si Dissemin devrait s'écarter d'ORCID en ne
l'utilisant plus comme système principal d'authentification, quitte à toujours
permettre l'authentification par ce biais et l'association d'un compte Dissemin
à un compte ORCID ; ou si Dissemin devrait formellement demander un accès
privilégié à l'API payante à ORCID avant de renoncer. Il est résolu de retenir
cette deuxième option : exposer nos griefs à ORCID, et préciser la solution
qu'on entend retenir si on ne peut pas avoir accès à l'API.

\paragraph{Réponse aux retours utilisateurs.}
Antoine soulève la question de répondre aux utilisateurs qui nous signalent des problèmes. Beaucoup de signalements concernent des erreurs dans les données de Dissemin, qui proviennent souvent de l'import de sources tierces. Ce problème serait réglé si nous avions une interface d'édition (prochain point), mais ce n'est pas encore le cas.

Il est décidé de répondre à ces courriels (et d'ajouter une entrèe à la FAQ) de la façon suivante : indiquer qu'on n'a pas encore d'interface d'édition, et expliquer que les données viennent de sources que nous ne contrôlons pas, et que les erreurs doivent être corrigées directement dans ces sources. Lorsque c'est possible, nous pourrions indiquer à l'utilisateur un exemple de source de donnée immédiate où l'erreur figure également ; quand l'investigation est plus complexe, on peut simplement signaler à l'utilisateur la liste des sources utilisées sans plus de précisions.

Antonin rappelle que ces problèmes seraient évités si Dissemin basculait entièrement vers des sources de données directement éditables, par exemple Wikicite.

\paragraph{Édition des données.}
L'assemblée juge que l'interface d'édition est une priorité importante de Dissemin, et que les premiers points (e.g., édition du titre) peuvent être faits sans difficultés majeures, même si d'autres (disponibilité, etc.) posent des questions plus complexes. Il est résolu de s'y attaquer lors des prochaines sessions de développement une fois réglés les problèmes techniques majeurs.

\paragraph{Campagne Wikimédia et statistiques.}
Federico rappelle l'action qu'il a entreprise, et qui a mené à environ 3000 dépôts dans Dissemin. Il indique que les principaux points d'achoppement dans l'utilisation de Dissemin sont :
    \begin{itemize}
    \item Faire en sorte que les dépôts n'échouent pas
    \item Simplifier, si possible, en ne nécessitant pas ORCID
\end{itemize}

CAPSH pourrait également aider :
    \begin{itemize}
    \item En lui venant en aide pour répondre aux courriels de réponse des chercheurs, ce qui est techniquement possible.
    \item En aidant au développement pour moissonner l'adresse de courriel de davantage de chercheurs.
    \item En persuadant certaines universités de contacter leurs chercheurs pour les inciter à déposer le texte intégral d'articles liés depuis Wikipédia. La possibilité de diffuser les statistiques de citation sur Wikipédia, ou de les mettre sur Wikidata, est évoquée.
    \end{itemize} 

\paragraph{Réunions périodiques et hackathons.}
L'assemblée reconnaît une valeur de coordination aux réunions mensuelles organisées l'année dernière, mais déplore que ces réunions nécessitent beaucoup de coordination et ne fassent pas vraiment avancer le projet. Patricia se dit prête à rédiger une lettre d'information mensuelle interne pour tenir l'équipe au courant des derniers développements, éliminant ainsi le besoin de la réunion.

Étant donné le succès du dernier hackathon, il est proposé de faire un événement de ce genre tous les trois mois environ, avec un événement public une fois par an. Il est rappelé que le budget PSL permet de financer des missions pour organiser ces hackathons. L'organisation pratique d'un hackathon est reportée à plus tard.

\bigskip

L'ordre du jour étant épuisé, la séance est levée à 22h56.
\end{document}

