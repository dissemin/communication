\documentclass[a4paper]{article}

\usepackage[official]{eurosym}
\usepackage[french]{babel}
\usepackage{hyperref}

\title{Procès-verbal de l'assemblée générale 2020}
\date{18 septembre 2019}

\begin{document}

\maketitle

L'assemblée générale ordinaire du Comité pour l'Accessibilité aux Publications en Sciences et Humanités (C.A.P.S.H.) s'est tenue le 18 septembre 2020 à 20 heures, sur la plateforme BigBlueButton du département INFRES de Télécom Paris. 

Présidence de séance: Marie Farge
Secrétaire de séance : Patricia Mirabile
Liste des présents : Antoine Amarilli, Antonin Delpeuch, Federico Leva, Marc Jeanmougin, Marie Farge, Pablo Rauzy (jusqu'au point 6), Patricia Mirabile, Charles Paperman (à partir du point 4), Lucas Verney (à partir du point 9).
    
\section{Bilan moral tenu par A. Delpeuch}

\begin{enumerate}
\item Fin officielle du projet OpenIng à la fin de l'année 2019, mais poursuite pendant 2020 en particulier sur l'authentification pour les membres d'autres institutions. Longue démarche pour obtenir un accès Shibboleth via Edugain, finalement fructueuse. 
\item No free view, no review lancé en mai/juin avec 292 signataires. Perspectives de communication et diffusion de cette campagne (mentions sur des sites web ? badges ? listes de diffusion ? affichages ?)
\item Projet avec F. Leva d'augmentation du nombre de références en OA dans les références d'articles Wikipedia. 
\item Q.: sur la promotion de dissemin étant donnée l'obligation maintenant généralisée de déposer sur HAL. Dissemin pourrait-il (techniquement) répondre à la demande maintenant renforcée d'une interface facilitée de dépôt dans HAL ? R.: Il reste à déterminer quelle interface (entre HAL et dissemin) propose la procédure de dépôt la plus aisée. Malheureusement, dissemin pour le moment n'est pas assez stable pour encourager le dépôt systématique via dissemin plutôt que via HAL. 
  Mise au vote: bilan moral approuvé à l'unanimité via scrutin électronique.
\end{enumerate}

\section{Bilan financier tenu par A. Amarilli}
Lors du dernier bilan financier à l'assemblée générale précédente, le 1er septembre 2019, notre compte en banque contenait 3165,46 EUR.

Dans la période du 1er septembre 2019 au 15 septembre 2020, les mouvements ci-dessous ont eu lieu.
\begin{itemize}
\item Revenus (dons) :
  \begin{itemize}
    \item Antoine Amarilli : 13 fois 5 EUR soit 65 EUR
    \item Pablo Rauzy : 12 fois 10 EUR soit 120 EUR
    \item Liberapay : 32.09 EUR de dons anonymes (versement via Stripe)
  \end{itemize}
\item Revenus (autres) :
\begin{itemize}
    \item TU Darmstadt 1811.40 EUR : paiement pour la contribution aux frais de la plateforme dans le cadre du projet OpenING
\end{itemize}
\item Dépenses
  \begin{itemize}
    \item Nom de domaine : 38.40 EUR et 10.80 EUR le 23/09, remboursement par chèque à Antoine Amarilli
    \item Hébergement auprès d'Online SAS : 13 fois 145.15 EUR soit 1886.95 EUR
  \end{itemize}
\end{itemize}

En date du 15 septembre 2020, notre compte en banque contient donc 3257.80 EUR, soit un bénéfice de 92.34 EUR sur la période.

En date du 15 septembre 2020, le compte de l'association sur Liberapay dispose de 0.03 EUR.

L'association a émis pour la première fois en avril 2020 des reçus fiscaux au titre des articles 200 et 238 bis du Code général des impôts. Ces reçus correspondent aux 12 dons de 5 EUR d'Antoine et aux 12 dons de 10 EUR de Pablo reçus pendant l'année 2019, soit un total de 180 EUR.

Les dépenses annuelles actuelles étant de l'ordre de 1800 EUR, la plateforme peut continuer à fonctionner jusqu'à fin 2021 environ sans dons, et jusqu'à l'été 2022 environ en comptant les dons récurrents.


Succès de la procédure de rescrit fiscal qui renforce les perspectives financières de l'association puisque la valeur des dons est maintenant démultipliée. La pérennité de l'association sur le court terme est assurée jusqu'à la fin de l'année prochaine, le long terme dépend du succès des candidatures aux appels à projets récents. Cette pérennité à long terme, si elle devient incertaine, devrait être interrogée aussi en termes de l'affluence des dépôts et des utilisateurs sur la plateforme. 
Notons que pour le moment, l'utilisation de la plateforme via le partenariat avec OpenIng reste équivalente à celle des autres utilisateurs, mais cela est sans doute dû à l'absence, jusqu'à récemment, de l'authentification via Shibboleth. 
Point d'information sur les conditions et réglementations concernant les dons et leurs défiscalisation. 
Mise au vote: bilan financier approuvé à l'unanimité via scrutin électronique.

\section{Renouvellement des membres du CA}
    Rappel de la liste des membres du CA et de l'association en général (total de 12 membres) et confirmation que toutes les personnes présentes sont membres de l'association.
    Proposition de promotion de Marc Jeanmougin du statut de membre associé au statut de membre actif. Hésitations de l'impétrant par syndrome de l'imposteur et de l'un des membres du CA pour motif de théorie des ensembles. 
    Mise au vote (ponctuée d'inquiétudes sur l'éligibilité des membres de l'association en tant que votants): approuvé à l'unanimité via scrutin électronique. Félicitations. 

    Mise au vote: renouvellement des membres du CA approuvé à l'unanimité via scrutin électronique.

Proposition de l'intronisation de Charles Paperman en tant que membre associé: approuvé à l'unanimité. 


\section{Mise à jour des statuts de l'association}
Aucune proposition de modification des statuts. 

\section{Financement PSL Explore}
Arrivée prochaine de la fin de validité de ce financement en 2021. Urgence donc d'épuiser les fonds pour le moment disponibles, par exemple au moyen de vacations. 
Tentative infructueuse de recrutement auprès des membres de l'association. Discussion de la possibilité d'extension de la durée de validité du recrutement. Discussion aussi de la possibilité d'utiliser les fonds pour la participation à des conférences.

Compte rendu de la participation de M. Farge au Barcamp à Berlin, qui s'est tenu en mars 2020, où elle a participé à un groupe de réflexion sur l'open diamond. Annonce aussi de l'ouverture de nouvelles plateformes d'open review. 

Proposition d'autres conférences: Open Repo 2021 pour présenter Opening (\url{https://or2020.sun.ac.za/2019/03/20/save-the-date-for-open-repositories-2020-1-4-june/})

Discussion ouverte sur les systèmes d'open reviewing et de modération des commentaires sur les systèmes d'archivages de preprints. 

\section{Rappel sur les candidatures en cours pour des appels à projets}
FNSO en attente, possibilité de recevoir jusqu'à 100 000  /200 000 euros. 
Inria engineer: demande portée par P. Senellart, mais réponse ralentie par la situation sanitaire actuelle.
ADT (?) aussi en attente. 

\section{Recrutement de développeurs}
Suggestion que des enseignants en école d'ingénieur pourraient faciliter le recrutement d'ingénieurs, potentiellement ceux qui se trouveraient parmi les membres de l'association. 
Évocation de l'incurie du système de recrutement dans l'ESR qui limite l'embauche d'ingénieurs qualifiés de façon tempestive. 
Possible voie de diffusion d'offres de poste via le système des RH de l'Inria.
Difficultés de recrutement d'un ingénieur bénévole sur la plateforme Dissemin : intérêt limité en dehors des bénéficiaires de la plateforme (les chercheurs) et existence d'une base de code dans un état d'avancement intermédiaire.
Tentative de recrutement du nouveau membre de l'association, infructueuse pour cause de manque de temps (et peut-être d'ingratitude).  

Point technique : Dissemin fonctionne bien sous Python 3. 

Commentaires techniques : grâce au travail de Stefan (avec OpenIng), l'intégration Crossref est maintenant fonctionnelle et rapide. 

Hypothèse du succès de ResearchGate en raison de son système de gamification et d'ingénierie sociale. 
Contre-point : le système de tentative de captation des chercheurs par ResearchGate n'est pas un modèle désirable pour dissemin (ou même en général). 
Contre-point supplémentaire : ResearchGate a pollué la conscience générale qui est déjà bien trop ignorante de ce qu'est l'Open Access. 
Complément : les chercheurs doivent garder leur copyright, et c'est leur disposition à perdre ce copyright qui a été historiquement l'une des sources de l'actuel système déplorable de publication scientifique. ResearchGate encourage encore plus cette disposition à renoncer au copyright.
Réponse : ResearchGate est un symptôme de la toxicité du système actuel de publication académique et peut constituer un compétiteur et repoussoir fertile. 

\section{Communiquer sur la différence entre APC et Open Access ?}
Rappel sur les différentes structures de publication et les différents modèles, ainsi que leur position dans le spectre de l'Open access.

Deux remarques : tout d'abord, la confusion trop répandue entre l'open access et les APC. Ensuite, le raccourci selon lequel sci-hub serait la solution gratuite pour l'accès aux publications scientifiques qui oublie le coût (moral, potentiellement financier) pour Alexandra Elbakyan. 

Proposition : pétition de soutien à sci-hub 
(Proposition rejetée fermement de pétition pour la disparition de sci-hub par assèchement du besoin qu'il résout).
Proposition délayée jusqu'à la stabilisation de Dissemin. 

Remarque : les compte-rendus de l'Académie des Sciences sont passés en Open Diamond. Mention de la science des fluides. 

Proposition : serait-il possible d'estimer le prix en APC des publications d'un chercheur ou d'une chercheuse données ? Réponse : obstacles techniques. 

Proposition : un sondage sur les montants payés en APC par les chercheurs. Inquiétudes sur la valeur épistémique des résultats d'une telle enquête étant donnée la probable disparition des APC (car deviendront incluses dans des "bundles" vendus aux universités). 

Encart : succès de la part de Marie dans la récupération de son copyright et la publication d'un copyright erratum par Cambridge University Press. 
Conclusion : déception constante et persistante à l'égard du système de publications scientifiques. 

Question : quelles formes devra prendre l'activisme open access lorsque la transition vers un modèle de bundles "APC / abonnements" sera accomplie ? 
Point-espoir : le prix de ces bundles ne sera-t-il pas tenu d'être rendu public ?
Contre-point-désespoir : non.

\section{Gestion des mails}
Faible taux de réponse aux e-mails reçus. Discussion de systèmes d'organisation pour améliorer la gestion des e-mails. Reportée en vertu du principe de \url{https://xkcd.com/1445/}.


Séance levée à 22h41. 


\end{document}

