\documentclass[a4paper]{article}

\usepackage[official]{eurosym}
\usepackage[french]{babel}
\usepackage{hyperref}

\title{Procès-verbal de l'assemblée générale 2021}
\date{20 septembre 2021}

\begin{document}

\maketitle

L'assemblée générale ordinaire du Comité pour l'Accessibilité aux Publications
en Sciences et Humanités (C.A.P.S.H.) s'est tenue le 20 septembre 2020 à 20 heures, sur la plateforme BigBlueButton du département INFRES de Télécom Paris. 

\begin{itemize}
  \item Présidence de séance : Pierre Senellart
  \item Secrétaire de séance : Antoine Amarilli
  \item Liste des présents : Antoine Amarilli, Antonin Delpeuch, Federico Leva, Marc Jeanmougin, Marie Farge, Pablo Rauzy, Pierre Senellart
\end{itemize}

\section{Ordre du jour}

\begin{itemize}
\item Bilan moral
\item Bilan financier
\item Élection du conseil d'administration
\item Statut de la procédure de recrutement d'un ingénieur à l'ENS
\item Migration de Dissemin de Online SAS vers Télécom Paris
\item Tâches restantes pour le projet PSL Explore
\end{itemize}

\section{Bilan moral}

Antonin Delpeuch présente le bilan moral de l'association (en anglais).

\begin{itemize}
  \item people continue to use Dissemin to deposit papers. From 2020-09-01 to 2021-09-01 we had 460 successful deposits, that is almost twice as much as the year before (2019-09-01 to 2020-09-01: 253 successful deposits).
  \item continued activity around the "No free view? No review!" campaign (almost 333 signatories to date!)
  \item ongoing improvements to Dissemin by Steph on the SHERPA/RoMEO integration, migrating from RoMEO v1 to RoMEO v2. Steph and Antonin are meeting regularly to coordinate the work.
  \item we migrated Dissemin out of GitHub to GitLab
  \item the OpenING project came to an end. We thank them very much (and in particular Stefan) for all the work they have done on Dissemin!
  \item the PSL Explore grant came to an end. We thank them for their valuable support for all these years!
  \item the VALDA team at ENS got funding from FNSO to work on Dissemin and is hiring a research engineer to that end
\end{itemize}

Il est confirmé que Stefan a pris un autre poste et ne contribue plus activement à la plateforme, même s'il est toujours le bienvenu.

Federico Leva mentionne le développement actif d'OAbot et son utilisation hebdomadaire sur la Wikipédia anglophone pour ajouter des liens en libre accès pour les articles identifiés par leur DOI. Ceci est rendu possible par le soutien d'Unpaywall, qui réduit les limites sur les requêtes. Il est prévu que ceci soit officiellement payé par Wikimedia Italia.

Il y a des problèmes de fiabilité sur la détection de si un PDF est en libre accès ou non. Libre accès "Bronze" : PDFs qui sont gratuitement téléchargeables, mais sans licence (couvre l'open access temporaire, après période d'embargo, etc.). Marie mentionne des problèmes de copyright avec Elsevier.

Le bilan moral est mis aux voix et il est approuvé à l'unanimité.

\section{Bilan financier}

Lors du dernier bilan financier à l'assemblée générale précédente, le 15
septembre 2020, notre compte en banque contenait 3257.80~EUR.

Dans la période du 15 septembre 2020 au 20 août 2021, les mouvements ci-dessous ont eu lieu.

\begin{itemize}
\item Revenus (dons) :
  \begin{itemize}
    \item Antoine Amarilli : 11 fois 5~EUR soit 55~EUR
    \item Pablo Rauzy : 3 fois 10~EUR soit 30~EUR
    \item Liberapay : 115.17~EUR de dons anonymes (versements via Stripe)
  \end{itemize}
\item Dépenses
  \begin{itemize}
    \item Nom de domaine : 38.40~EUR et 10.80~EUR, remboursement par chèque le
      12/10 à Antoine Amarilli
    \item Hébergement auprès d'Online SAS : 11 fois 145.15~EUR soit un total de 1596.65~EUR
    \item Banque : 21~EUR pour ouverture d'un compte chez Wise, et 0.35~EUR pour
      frais de virement
  \end{itemize}
\end{itemize}

En date du 20 août 2021, notre compte en banque contient donc 1790.77~EUR, soit
un déficit de 1467.03~EUR sur la période.

En raison de l'augmentation des frais de notre banque le Crédit Agricole au
premier janvier 2021, et d'incidents multiples de prélèvements SEPA imputés à
tort sur le compte, l'association a ouvert un nouveau compte en banque auprès de
Wise le 11 décembre 2020.

En date du 22 août 2021, le compte de l'association sur Liberapay dispose de
1.14~EUR, et a deux donneurs anonymes actifs pour un total de 0.51~EUR par
semaine.

L'association a émis en 2021 des reçus fiscaux au titre des articles 200 et 238
bis du Code général des impôts. Ces reçus correspondent aux 12 dons de 5~EUR
d'Antoine et aux 11 dons de 10~EUR de Pablo reçus pendant l'année 2020, soit un total de 170~EUR.

Les dépenses annuelles actuelles étant de l'ordre de 1800~EUR, la plateforme
peut continuer à fonctionner jusqu'à l'été 2022 environ sans dons, L'assemblée
générale doit donc se prononcer sur la marche à suivre.

L'association a candidaté à l'appel à projet du Fonds national pour la science
ouverte 2019 pour un montant de 124\,200~EUR, dont l'acceptation a été notifiée
début novembre 2020. L'emploi principal prévu pour cet argent est de recruter un
ingénieur pendant deux ans. Ces fonds sont gérés par l'ENS.

Le bilan financier est mis aux voix. Il est approuvé à l'unanimité.

\section{Élection du conseil d'administration}

Antoine Amarilli propose de reconduire les membres actuels du conseil d'administration dans leurs fonctions. Cette motion est approuvée à l'unanimité. L'assemblée générale ne propose pas l'introduction d'un nouveau membre au conseil d'administration.

\section{Statut de la procédure de recrutement d'un ingénieur à l'ENS}

Pierre Senellart fait l'état des lieux du recrutement en cours, de la sélection d'un candidat, et de la rémunération qui peut lui être proposée.

Le candidat au recrutement attend de recevoir un passeport américain pour pouvoir voyager. Les ressources humaines de l'ENS ont fait le nécessaire pour permettre au candidat de travailler en France. L'espoir initial était que le candidat puisse commencer au premier novembre, mais cela peut être repoussé, en fonction de si l'autorisation de travail est approuvée.

Pierre mentionne un autre candidat envisageable.

Il n'y a pas encore de problème par rapport à la durée du projet.

En cas d'échec du recrutement, on pourrait demander de convertir cet argent pour payer des prestataires, mais ce ne serait probablement pas une bonne idée pour un travail suivi.

\section{Migration de Dissemin de Online SAS vers Télécom Paris}

L'assemblée générale fait le point sur les ressources actuellement louées par
l'association.
Il y a une machine avec un disque suffisamment rapide (SSD) pour une grosse base
de données relationnelle, actuellement 500 GB en RAID1, et également la machine
principale qui a un CPU Intel(R) Atom(TM) CPU  C2750  @ 2.40GHz. Au total, il y
a quatre machines.

Marc Jeanmougin fait le point sur ce que Télécom Paris et le département INFRES
pourraient envisager, notamment l'hébergement d'une machine achetée pour le
projet.

Pour financer cela, outre les fonds de l'association, il y a 2000 EUR de
consommables et matériel sur le projet FNSO, mais déjà dépensé pour la machine
de l'ingénieur, et 5000 EUR de frais de prestation.

Les modes de paiement proposés par l'hébergeur Online sont le retrait SEPA, la
carte bancaire, et Paypal.

Il est convenu que Pierre clarifie avec l'ENS quels modes de paiement sont
possibles pour l'hébergement des services du projet, éventuellement via
l'association, pour que les frais d'hébergement soient pris en charge par
l'association. La discussion sur un ré-hébergement du projet à Télécom est
repoussée à la prise de fonctions de l'ingénieur.

\section{Tâches restantes pour le projet PSL Explore}

Marie et Antonin ont écrit un rapport final pour PSL Explore, qui reçoit
l'approbation de l'assemblée générale.

\section{Autres points}

Marie mentionne son expérience à la conférence Open Science et la prochaine
conférence Open Science.

Une discussion s'engage sur les orientations futures du projet.

\bigskip

L'ordre du jour étant épuisé, la séance est levée à 21h55.

\end{document}
