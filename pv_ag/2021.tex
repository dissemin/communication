\documentclass[a4paper]{article}

\usepackage[official]{eurosym}
\usepackage[french]{babel}
\usepackage{hyperref}

\title{Procès-verbal de l'assemblée générale 2021}
\date{20 septembre 2021}

\begin{document}

\maketitle

L'assemblée générale ordinaire du Comité pour l'Accessibilité aux Publications
en Sciences et Humanités (C.A.P.S.H.) s'est tenue le 20 septembre 2020 à 20 heures, sur la plateforme BigBlueButton du département INFRES de Télécom Paris. 

Présidence de séance : TODO
Secrétaire de séance : TODO
Liste des présents : TODO
    
\section{Bilan moral tenu par A. Delpeuch}

TODO

\section{Bilan financier tenu par A. Amarilli}
Lors du dernier bilan financier à l'assemblée générale précédente, le 15
septembre 2020, notre compte en banque contenait 3257.80~EUR.

Dans la période du 15 septembre 2020 au 20 août 2021, les mouvements ci-dessous ont eu lieu.

\begin{itemize}
\item Revenus (dons) :
  \begin{itemize}
    \item Antoine Amarilli : 11 fois 5~EUR soit 55~EUR
    \item Pablo Rauzy : 3 fois 10~EUR soit 30~EUR
    \item Liberapay : 115.17~EUR de dons anonymes (versements via Stripe)
  \end{itemize}
\item Dépenses
  \begin{itemize}
    \item Nom de domaine : 38.40~EUR et 10.80~EUR, remboursement par chèque le
      12/10 à Antoine Amarilli
    \item Hébergement auprès d'Online SAS : 11 fois 145.15~EUR soit 1596.65~EUR
    \item Banque : 21~EUR pour ouverture d'un compte chez Wise, et 0.35~EUR pour
      frais de virement
    \item 
  \end{itemize}
\end{itemize}

En date du 20 août 2021, notre compte en banque contient donc 1790.77~EUR, soit
un déficit de 1467.03~EUR sur la période.

En raison de l'augmentation des frais de notre banque le Crédit Agricole au
premier janvier 2021, et d'incidents multiples de prélèvements SEPA imputés à
tort sur le compte, l'association a ouvert un nouveau compte en banque auprès de
Wise le 11 décembre 2020.

En date du 22 août 2021, le compte de l'association sur Liberapay dispose de
1.14~EUR, et a deux donneurs anonymes actifs pour un total de 0.51~EUR par
semaine.

L'association a émis en 2021 des reçus fiscaux au titre des articles 200 et 238
bis du Code général des impôts. Ces reçus correspondent aux 12 dons de 5~EUR
d'Antoine et aux 11 dons de 10~EUR de Pablo reçus pendant l'année 2020, soit un total de 170~EUR.

Les dépenses annuelles actuelles étant de l'ordre de 1800~EUR, la plateforme
peut continuer à fonctionner jusqu'à l'été 2022 environ sans dons, L'assemblée
générale doit donc se prononcer sur la marche à suivre

L'association a candidaté à l'appel à projet du Fonds national pour la science
ouverte 2019 pour un montant de 124\,200~EUR, dont l'acceptation a été notifiée
début novembre 2020. L'emploi principal prévu pour cet argent est de recruter un
ingénieur pendant deux ans. Ces fonds sont gérés par l'ENS.

\section{Renouvellement des membres du CA}
TODO

\section{Other points}
TODO

\end{document}

